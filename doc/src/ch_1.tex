\section{\S 1}

\subsection{gccでコンパイル}

gccでコンパイルする時のオプション

\begin{itemize}

   \item -Wall, 警告をできるだけ多く出すオプション

   \item -O1, -O2, -O3 最適化オプション

\end{itemize}

\subsection{コマンドライン引数}

main関数の引数 int argcm, char* argv[]があるが、
argv[0]は実行されるプログラム本体で、argv[1],argv[2],..,argv[i]はそれぞれ1,2,..,i番目のコマンドライン変数である。 

\section{\S 2}

\subsection{Linuxの世界}

Linuxはファイルシステム、プロセス、システムコールで構成される。

システムはカーネルとシェルで分割される。

カーネルの仕事 -> ハードウェアを操作、カーネル外からハードウェアを操作するためにはカーネルに命令を出す。
この命令をシステムコールという。システムコールには以下のようなものがある。

\begin{itemize}
  
  \item open
  \item read
  \item write
  \item fork
  \item exec
  \item stat
  \item unlik 

\end{itemize}

\subsection{システムコール、ライブラリ、API}

システムコールが一番低いレイヤーの命令でその命令を使って使いやすい関数が実装されている。
例としてはprintf()関数があるが内部ではシステムコールwrite()関数を呼び出している。ただシステムコールとライブラリー関数の
違いは曖昧である。このライブラリー関数はLinuxではGNU libc (glibc)にて実装されている。

API(Application Programing interface)とは何かを利用してプログラミングを行う際の用いるインターフェースでる。
LinuxプログラミングならばCのライブラリ関数が該当する。
